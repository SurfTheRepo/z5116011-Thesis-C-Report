\chapter{Introduction}\label{ch:intro}

Since the release of bitcoin in 2009 and its immense surge in popularity in the years after, blockchain has become one of the most talked about topics in both business and research realms. This has resulted in many people trying to replicate the success of bitcoin and the underlying technology. The initial focus of the application of blockchain solutions has been largely dominated by cryptocurrencies, but with constant improvements in the technology other applications are becoming available. This is mainly due to the addition of smart contracts and private networks to the blockchain technology and the increasing computing power of devices and reduction of resources consumed by blockchain technology. These improvements have allowed organisations to start adopting blockchain technologies in their business practices, such as managing internal and external transfer of assets, replacing traditional databases.

Internet of Things (IoT) devices have rapidly improved in their ability in recent years, with their size decreasing, portability increasing and general computational power on the rise, allowing for their application to realised in many fields. These devices allow for the use of digital solutions in areas where cost and other resource constraints provide a barrier to participation. Using IoT devices to run a blockchain network is an important part of furthering the blockchain technology, bringing blockchain solutions to industries previously incapable of implementing them.

Together these technologies have the potential to revolutionise several industries, such as supply chains and construction management. Due to the meteoric rise of blockchain, concerns of stability, secureness, practicality and portability are some the reasons industries have been hesitant to adopt this technology on non resource constrained devices, let alone low powered IoT devices which introduce further concerns to the possible practicality.

Before industries would accept running a blockchain solution on a network of low powered powered IoT devices they would require a proof of concept. Running a blockchain solution that is designed for industries, such as Hyperledger Fabric, on a network of IoT devices, such as a Raspberry Pi, is the aim of this thesis.

In order to understand the potential and limitations of such a network, monitoring this network will be essential. This will allow for data to be gathered on the effectiveness of the network, monitoring both blockchain network and device specific metrics.
